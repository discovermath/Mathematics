\documentclass{article}
\usepackage[utf8]{inputenc}


\title{The Exponential Inequality}
\author{Saksham Sethi}
\date{June 2021}

\begin{document}

\maketitle
\noindent\makebox[\linewidth]{\rule{\paperwidth}{0.8pt}}

\section{A case of the generalized form}
$n^4 \le 4^n$ for all $n \ge 4$.\\


\noindent\makebox[\linewidth]{\rule{\paperwidth}{0.8pt}}
\section{Proof of case}
We try to prove this using induction. \\



\textbf{Base Case: $n=4$:} $4^4 \le 4^4$. This works. 
\\




\textbf{Inductive step:}  Let $k$ satisfy this statement, making $k^4 \le 4^k$. We have to prove that $(k+1)^4 \le 4^{k+1}$. Multiplying both sides of the first equation by $4$, we get $4k^4 \le 4^{k+1}$. So it suffices to show that $(k+1)^4 \le 4k^4$. Since both sides are positive, quad-root and get $(k+1) \le \sqrt{2}k$. This holds if and only if $1 \le (\sqrt{2}-1)k$, which holds true if and only if$k \ge \frac{1}{\sqrt{2}-1} = \frac{1}{\sqrt{2}-1} \times \frac{\sqrt{2}+1}{\sqrt{2}+1} = \sqrt{2}+1$. Because $k\ge4$, we know that $k\ge \sqrt{2}+1$ as well since $\sqrt{2}+1 < 4$. All our steps are reversible, so we can just walk backwards to conclude that $(k+1)^4 \le 4^{k+1}$. \\

Our induction is complete. 

\noindent\makebox[\linewidth]{\rule{\paperwidth}{0.8pt}}


\section{Extension}
\subsection{Statement excluding one case (has a missing constraint)}
$n^k \le k^n$ for all $n\ge k$, where $n$ and $k$ are positive integers. 

\subsection{Proof}

We use induction again, except this time, the induction is on the varying values of $k$. The base case is going to be 2. This uses "double induction". \\



\textbf{Base Case on the values of $n$: $n=3$: } $n^3 \le 3^n$ for all $n \ge 3$\\


	
 \textbf{Base Case on the values of $k$ when $n=3$:} $3^3 \le 3^3$ works. \\
        
        
        
 \textbf{Induction on the values of $k$ when $n=3$:} We can trace our path from the case we took of the generalized form. \\ \\
        
        

\textit{We've already proved that it the general form works when $n=3$, so now we just have to prove using induction that if it works for an positive integer $m$, it will work for $m+1$.} \\ \\ 




\textbf{Induction on the values of $n$:} Let $n$ work, which means $n^k \le k^n$. We now have to prove $(n+1)^k \le k^{n+1}$. Multiply both sides of the first inequality by $k$ to get $kn^k \le k^{n+1}$. So it suffices to show that $(n+1)^k \le kn^k$. Since both sides are positive, $k\text{-root}$ both sides to get $(n+1) \le kn$. This holds if and only if $1 \le n(k-1)$ is true, which is true if and only if $\frac{1}{n} \le k-1$. If we recall, $n \ge k$ and $n$ and $k$ are positive integers. This makes $0 \le \frac{1}{n} \le 1$ and $k-1=0, 1, 2, 3, 4, ...$ .    If $k-1=1$, and $\frac{1}{n}$ is at maximum $1$, the constraint is satisfied. If $k-1>1$, the constraint is satisfied as well. Now, if $k-1=0$, then $k=1$ and the original inequality becomes $n^1 \le 1^n$ which is $n \le 1$. This isn't necessarily true; in fact, it is never true. Thus we have to fix our statement and make $k>1$. Other than that, we can reverse all our steps from bottom to up and complete the induction. 
\noindent\makebox[\linewidth]{\rule{\paperwidth}{0.8pt}}

\section{Final fixed statement}
$n^k \le k^n$ for all $n\ge k$, where $n$ and $k$ are positive integers, and $k \ge 1$. \square

\noindent\makebox[\linewidth]{\rule{\paperwidth}{0.8pt}}\\

\end{document}
